\documentclass[12pt]{article}
\usepackage[T1]{fontenc}
\usepackage{lmodern}
\usepackage{graphicx}
\usepackage[hidelinks]{hyperref}
\usepackage{float}
\usepackage{tcolorbox}
\usepackage[margin=1.25in]{geometry}

\newcommand{\executable}[1]{{\fontfamily{pcr}\selectfont\textbf{#1}}}
\newcommand{\parameters}[1]{{\fontfamily{pcr}\selectfont#1}}

\newcommand{\hrefc}[2]{\href{#1}{\color{blue}#2}}

% \setlength{\textwidth}{6.25in}
\setlength\parindent{0pt}

% \graphicspath{ {./images/} }

\makeindex

\title{ClC\_MKM v0.1 Manual}
\author{Austen Bernardi}
\date{\today}

\begin{document}
\maketitle{}

\tableofcontents
\clearpage
\newpage

\section{Introduction}

ClC\_MKM contains Python libraries and executables for steady-state kinetic modeling and optimization a ClC-ec1 Markov State Model. ClC-ec1 is a secondary-active Chloride/Proton transmembrane antiporter \cite{mayes2018multiscale}. ClC\_MKM supports optimization of kinetic rate coefficients between biologically relevant transitions against experimentally derived unitary turnover rates \cite{lim2009intracellular}. This manual describes how to use ClC\_MKM to perform optimization.

\section{License}
ClC\_MKM uses the GNU General Public License v3.  See \parameters{ClC\_MKM/LICENSE} for more information.

\section{Requirements}
ClC\_MKM depends on the following requirements:

\begin{itemize}
\item Python3 (python3)
\item SciPy (python package)
\item SpotPy (python package)
\item NumPy (python package)
\end{itemize}

\section{Installation}
Download the top directory \parameters{ClC\_MKM} to a directory that will store the module. Update your \parameters{\$PYTHONPATH} environment variable to include the storage directory if it is not already included. Update your \parameters{\$PATH} environment variable to include \parameters{ClC\_MKM/bin}. For example, on a Unix based system, if the storage directory is \parameters{\$HOME/modules}, then adding \\

\parameters{export PYTHONPATH=\$PYTHONPATH:"\$HOME/modules"}

\parameters{export PATH=\$PATH:"\$HOME/modules/ClC\_MKM/bin"}\\

to \parameters{\$HOME/.bashrc} properly installs ClC\_MKM.

\clearpage
\section{Executables}
\label{sec:exec}

This section describes the executables provided in \parameters{ClC\_MKM/bin}.

\subsection{\executable{run\_opt.py}}
\executable{run\_opt.py }\parameters{<config>}\\[0.2cm]
\textbf{Parameters}\\
\parameters{<config>}: A configuration file containing parameters for the ClC-ec1 system.  Also specifies the optimization configuration file.  The format of this configuration file is specified in section \ref{sec:sys_config}. \\[0.2cm]
\textbf{Description}\\
Performs optimization on the ClC-ec1 system specified by \parameters{<config>}. See section \ref{sec:examp} for a usage examples.\\[0.2cm]

\subsection{\executable{run\_mkm.py}}
\executable{run\_mkm.py }\parameters{<config>}\\[0.2cm]
\textbf{Parameters}\\
\parameters{<config>}: A configuration file containing parameters for the ClC-ec1 system.  The format of this configuration file is specified in section \ref{sec:config}. \\[0.2cm]
\textbf{Description}\\
Generates an instance of the ClC-ec1 system specified by \parameters{<config>}. See section \ref{sec:examp} for a usage example.\\[0.2cm]

\section{Configuration files}
\label{sec:config}
This section describes the various configuration files used by ClC\_MKM.  Representative example configuration files are provided in the \parameters{ClC\_MKM/examples} directory, and discussed in section \ref{sec:examp}.  Lines may be commented using the '\#' symbol. The parameter specification format for all configuration files is \parameters{param = <param\_list>}, where \parameters{<param\_list>} is a comma-seperated (without spaces) list of values, ex. \parameters{<val1,val2,val3,...>}.  Space seperated text after \parameters{<param\_list>} is ignored and can be used as comments (ex. units).  If any \parameters{<param\_list>} has length greater than one, all other \parameters{<param\_list>} must have the same length or length one.  If a \parameters{<param\_list>} is length one and another \parameters{<param\_list>} has length greater than one, then the value is used for all systems. In some cases, \parameters{<param\_list>} must be length one, indicated by \parameters{scalar}. A listed \parameters{param} is considered to be required unless specified as \parameters{optional}.  

\subsection{Systems configuration file}
\label{sec:sys_config}
This is the main configuration file that is used as the argument for both executables listed in section \ref{sec:exec}.  See below for the full list of accepted parameters.\\

\executable{input\_rate\_file} \parameters{(scalar)}

The input rate coefficient filename for the kinetic model. File should be comma-seperated.  Line 1: parameter identifiers. Line 2: parameter values. Line 3: lower bounds for optimization. Line 4: upper bounds for optimization.  If lines 3 and 4 are not present, lower and upper bounds are uniformly set to zero and infinity, respectively.  See \parameters{ClC\_MKM/examples/opt/custom/seed\_dru\_san.csv} for an example with bounds specified. Units are 1/ms.\\

\executable{rate\_map\_file} \parameters{(scalar)}

The rate coefficient map filename for the kinetic model.  Used to map coefficients to applicable transitions.  See examples for formatting.\\

\executable{internal\_pH}

The pH(s) inside the vesicles of the modeled systems.\\

\executable{external\_pH}

The pH(s) outside the vesicles of the modeled systems.\\

\executable{internal\_Cl\_conc}

The chloride concentration(s) in mol/m$^3$ inside the vesicles.\\

\executable{external\_Cl\_conc}

The chloride concentration(s) in mol/m$^3$ outside the vesicles.\\

\executable{enzyme\_MW}

The molecular weight of the antiporter in g/mol.\\

\executable{lipid\_MW}

The molecular weight of the lipids that make up the vesicles in g/mol.\\

\executable{area\_per\_lipid}

The average surface area per lipid in m$^2$ of the lipids that make up the vesicles.\\

\executable{enzyme\_lipid\_wtfrac}

The weight fraction of enymes to lipids of the vesicles.\\

\executable{h\_rxn\_bl}

The approximate height of the reactive boundary layer for vesicle surface uptake reactions.\\

\executable{diffusivity\_Cl}

The bulk diffusivity of Chlorides.\\

\executable{diffusivity\_H}

The bulk diffusivity in m$^2$/ms of protons.\\

\executable{vesicle\_diam}

The average diameter in m of the vesicles.\\

\executable{enzyme\_surf\_conc\_sim}

The surface concentration of enzymes in mol/m$^2$ for the simulations used to model the uptake coefficients.\\

\executable{opt\_config\_file} \parameters{(scalar)}

Required only for \executable{run\_opt.py}. The filename of the optimization configuration file. See section \ref{sec:opt_config} for details.\\

\subsection{Optimization configuration file}
\label{sec:opt_config}
This configuration file is specified by the main configuration file as \parameters{opt\_config\_file}. For use with \executable{run\_opt.py}.  See below for the full list of accepted parameters.\\

\executable{opt\_package} \parameters{(optional, scalar)}

The name of the optimization package to use.  A custom combined steepest descent/conjugate gradient method is used if unspecified.  Supported packages are "scipy" and "spotpy". See examples \ref{} for example use cases.\\

\executable{opt\_residuals\_file} \parameters{(scalar)}

The optimization residuals filename.  This file contains residual targets for specified flows, and is used to build the objective function using a sum of square residual differences. See section \ref{sec:res_config} for details.\\

\executable{opt\_dat\_file} \parameters{(optional,scalar)}

Filename for the output optimization data (step, parameters, objective). Default value is \parameters{"opt.dat"}.\\

\executable{n\_steps} \parameters{(scalar)}

The number of steps for optimization (outermost level).\\

\executable{output\_interval} \parameters{(scalar)}

The interval between consecutive output records. A value of 1 records every step, a value of 2 records every other step, etc.\\

\executable{local\_method} \parameters{(scalar)}

Only used when \parameters{opt\_package} is "scipy". The local optimization method. Accepted values are listed under the "method" parameter of the \hrefc{https://docs.scipy.org/doc/scipy/reference/reference/generated/scipy.optimize.minimize.html}{\parameters{scipy.optimize.minimize}} documentation.\\

\executable{local\_options\_file} \parameters{(optional, scalar)}

Only used when \parameters{opt\_package} is "scipy". The local optimization options configuration filename. Format is consistent with the generic configuration file format specified in section \ref{sec:config}.  Only supports scalar parameters. Accepted values are listed are consistent with the arguments listed under the specific SciPy \hrefc{https://docs.scipy.org/doc/scipy/reference/reference/optimize.html\#local-multivariate-optimization}{local method}  documentation, with exception to the arguments \parameters{maxiter} and \parameters{bounds}, which are automatically specified by ClC\_MKM. See examples \ref{} for an example use case.\\

\executable{global\_method} \parameters{(optional,scalar)}

Only used when \parameters{opt\_package} is "scipy". The global optimization method. Accepted values are listed under the \hrefc{https://docs.scipy.org/doc/scipy/reference/reference/optimize.html\#global-optimization}{Global Optimization} section of SciPy's optimize documentation. Method "brute" is not supported.\\

\executable{global\_options\_file} \parameters{(optional, scalar)}

Only used when \parameters{opt\_package} is "scipy". The global optimization options configuration filename. Format is consistent with the generic configuration file format specified in section \ref{sec:config}.  Only supports scalar parameters. Accepted values are listed are consistent with the arguments listed under the specific SciPy \hrefc{https://docs.scipy.org/doc/scipy/reference/reference/optimize.html\#global-optimization}{global method}  documentation, with exception to the arguments \parameters{niter/maxiter} and \parameters{bounds}, which are automatically specified by ClC\_MKM. See examples \ref{} for an example use case.\\

\executable{algorithm} \parameters{(scalar)}

Only used when \parameters{opt\_package} is "spotpy".  Specifies the algorithm used for SpotPy optimization.  See SpotPy's \hrefc{https://spotpy.readthedocs.io/en/latest/Algorithm_guide/}{Algorithm Guide} for a list of accepted values (lowercase abbreviations).\\

\executable{limp} \parameters{(scalar)}

Only used when \parameters{opt\_package} is not specified.  Specifies the target lower bound for improvement in the objective for a single step.\\

\executable{uimp} \parameters{(scalar)}

Only used when \parameters{opt\_package} is not specified.  Specifies the target upper bound for improvement in the objective for a single step.\\

\executable{max\_limp\_steps} \parameters{(scalar)}

Only used when \parameters{opt\_package} is not specified.  Specifies the maximum number of steps in which the improvement is below \parameters{limp} before switching from steepest descent to conjugate gradient.\\

\subsection{Residual configuration file}
\label{sec:res_config}

The residual configuration file details ion flow targets for optimization, which are combined using sum of squared residual differences.  Accepted parameters are listed below.  Note the length of \parameters{<param\_list>} must be consistent with the number of systems specified in the main configuration file, following the \parameters{<param\_list>} rules specified in section \ref{sec:config}. If a value is specified as \parameters{NaN}, then the corrosponding residual flow is ommitted from the objective function calculation.\\

\executable{net\_Cl\_flow} \parameters{(optional)}

The net chloride flow(s) directed from external to internal in ions/ms per enzyme.\\

\executable{net\_H\_flow} \parameters{(optional)}

The net proton flow(s) directed from external to internal in ions/ms per enzyme.\\

\executable{bio\_Cl\_flow} \parameters{(optional)}

The chloride flow(s) directed from external to internal in ions/ms per enzyme for biologically oriented enzymes.\\

\executable{bio\_H\_flow} \parameters{(optional)}

The proton flow(s) directed from external to internal in ions/ms per enzyme for biologically oriented enzymes.\\

\executable{opp\_Cl\_flow} \parameters{(optional)}

The chloride flow(s) directed from external to internal in ions/ms per enzyme for oppositely oriented enzymes.\\

\executable{opp\_H\_flow} \parameters{(optional)}

The proton flow(s) directed from external to internal in ions/ms per enzyme for oppositely oriented enzymes.\\

\executable{no\_flow\_sys} \parameters{(optional)}

Special boolean residual parameter ("True" or "False"). Designates a system to have no flow anywhere (microscopic reversibility), generally for use with zero gradient boundary conditions. See \parameters{ClC\_MKM/examples/opt/custom/opt\_residuals.txt} for a usage example.\\

\section{Examples}
\label{sec:examp}
See \parameters{ClC\_MKM/examples} for some usage examples.  \parameters{ClC\_MKM/examples/mkm} provides a single instance of the ClC-ec1 kinetic system to be executed with \parameters{"run\_mkm.py config.txt"}. All other examples are designed to be executed within their respective directories with \parameters{"run\_opt.py config.txt"}.

\section{References}
\begingroup
\renewcommand{\section}[2]{}
\bibliographystyle{plain}
\bibliography{the_bibliography.bib}
\endgroup
\end{document}